% Appendix A

\chapter{Institutional Brokers' Estimate System} % Main appendix title

\label{AppendixA} % For referencing this appendix elsewhere, use \ref{AppendixA}

\lhead{\emph{I/B/E/S}} % This is for the header on each page - perhaps a shortened title

The Institutional Brokers' Estimate System (I/B/E/S) is a service founded by the New York brokerage firm Lynch, Jones and Ryan and Technimetrics, Inc. I/B/E/S began collecting earnings estimates for U.S. companies around 1976 and used the raw data to calculate statistical time series for each company. The data subsequently was used as the basis for articles in academic finance journals attempting to demonstrate that changes in consensus earnings estimates could identify opportunities to capture excess returns in subsequent periods. After starting with annual earnings estimates and estimates of "Long Term Growth, the database later was expanded to include quarterly earnings estimates. This allowed for the analysis of "Quarterly Earnings Surprises." Other innovations made possible by the I/B/E/S data included estimates for various equity indexes on a "top down" basis (made by strategists and economists) and estimates made on a "bottom up" basis (by individual analysts) for those same indexes. In the mid-1980s I/B/E/S began to expand its dataset to include companies trading in international markets. Lynch, Jones was sold to Citigroup in 1986. Barra bought I/B/E/S in 1993, selling it to Primark two years later. Thomson Financial purchased Primark in 2000. Successor companyThomson Reuters spun off its financial division under the name Refinitiv in 2018.

The I/B/E/S database currently covers over 40,000 companies in 70 markets. It provides to a client base of 50,000 institutional money managers. More than 900 firms contribute data to I/B/E/S, from the largest global houses to regional and local brokers, with US data back to 1976 and international data back to 1987.